\PassOptionsToPackage{unicode=true}{hyperref} % options for packages loaded elsewhere
\PassOptionsToPackage{hyphens}{url}
%
\documentclass[]{book}
\usepackage{lmodern}
\usepackage{amssymb,amsmath}
\usepackage{ifxetex,ifluatex}
\usepackage{fixltx2e} % provides \textsubscript
\ifnum 0\ifxetex 1\fi\ifluatex 1\fi=0 % if pdftex
  \usepackage[T1]{fontenc}
  \usepackage[utf8]{inputenc}
  \usepackage{textcomp} % provides euro and other symbols
\else % if luatex or xelatex
  \usepackage{unicode-math}
  \defaultfontfeatures{Ligatures=TeX,Scale=MatchLowercase}
\fi
% use upquote if available, for straight quotes in verbatim environments
\IfFileExists{upquote.sty}{\usepackage{upquote}}{}
% use microtype if available
\IfFileExists{microtype.sty}{%
\usepackage[]{microtype}
\UseMicrotypeSet[protrusion]{basicmath} % disable protrusion for tt fonts
}{}
\IfFileExists{parskip.sty}{%
\usepackage{parskip}
}{% else
\setlength{\parindent}{0pt}
\setlength{\parskip}{6pt plus 2pt minus 1pt}
}
\usepackage{hyperref}
\hypersetup{
            pdftitle={CorrelAid Data Journey},
            pdfauthor={Yihui Xie},
            pdfborder={0 0 0},
            breaklinks=true}
\urlstyle{same}  % don't use monospace font for urls
\usepackage{color}
\usepackage{fancyvrb}
\newcommand{\VerbBar}{|}
\newcommand{\VERB}{\Verb[commandchars=\\\{\}]}
\DefineVerbatimEnvironment{Highlighting}{Verbatim}{commandchars=\\\{\}}
% Add ',fontsize=\small' for more characters per line
\usepackage{framed}
\definecolor{shadecolor}{RGB}{248,248,248}
\newenvironment{Shaded}{\begin{snugshade}}{\end{snugshade}}
\newcommand{\AlertTok}[1]{\textcolor[rgb]{0.94,0.16,0.16}{#1}}
\newcommand{\AnnotationTok}[1]{\textcolor[rgb]{0.56,0.35,0.01}{\textbf{\textit{#1}}}}
\newcommand{\AttributeTok}[1]{\textcolor[rgb]{0.77,0.63,0.00}{#1}}
\newcommand{\BaseNTok}[1]{\textcolor[rgb]{0.00,0.00,0.81}{#1}}
\newcommand{\BuiltInTok}[1]{#1}
\newcommand{\CharTok}[1]{\textcolor[rgb]{0.31,0.60,0.02}{#1}}
\newcommand{\CommentTok}[1]{\textcolor[rgb]{0.56,0.35,0.01}{\textit{#1}}}
\newcommand{\CommentVarTok}[1]{\textcolor[rgb]{0.56,0.35,0.01}{\textbf{\textit{#1}}}}
\newcommand{\ConstantTok}[1]{\textcolor[rgb]{0.00,0.00,0.00}{#1}}
\newcommand{\ControlFlowTok}[1]{\textcolor[rgb]{0.13,0.29,0.53}{\textbf{#1}}}
\newcommand{\DataTypeTok}[1]{\textcolor[rgb]{0.13,0.29,0.53}{#1}}
\newcommand{\DecValTok}[1]{\textcolor[rgb]{0.00,0.00,0.81}{#1}}
\newcommand{\DocumentationTok}[1]{\textcolor[rgb]{0.56,0.35,0.01}{\textbf{\textit{#1}}}}
\newcommand{\ErrorTok}[1]{\textcolor[rgb]{0.64,0.00,0.00}{\textbf{#1}}}
\newcommand{\ExtensionTok}[1]{#1}
\newcommand{\FloatTok}[1]{\textcolor[rgb]{0.00,0.00,0.81}{#1}}
\newcommand{\FunctionTok}[1]{\textcolor[rgb]{0.00,0.00,0.00}{#1}}
\newcommand{\ImportTok}[1]{#1}
\newcommand{\InformationTok}[1]{\textcolor[rgb]{0.56,0.35,0.01}{\textbf{\textit{#1}}}}
\newcommand{\KeywordTok}[1]{\textcolor[rgb]{0.13,0.29,0.53}{\textbf{#1}}}
\newcommand{\NormalTok}[1]{#1}
\newcommand{\OperatorTok}[1]{\textcolor[rgb]{0.81,0.36,0.00}{\textbf{#1}}}
\newcommand{\OtherTok}[1]{\textcolor[rgb]{0.56,0.35,0.01}{#1}}
\newcommand{\PreprocessorTok}[1]{\textcolor[rgb]{0.56,0.35,0.01}{\textit{#1}}}
\newcommand{\RegionMarkerTok}[1]{#1}
\newcommand{\SpecialCharTok}[1]{\textcolor[rgb]{0.00,0.00,0.00}{#1}}
\newcommand{\SpecialStringTok}[1]{\textcolor[rgb]{0.31,0.60,0.02}{#1}}
\newcommand{\StringTok}[1]{\textcolor[rgb]{0.31,0.60,0.02}{#1}}
\newcommand{\VariableTok}[1]{\textcolor[rgb]{0.00,0.00,0.00}{#1}}
\newcommand{\VerbatimStringTok}[1]{\textcolor[rgb]{0.31,0.60,0.02}{#1}}
\newcommand{\WarningTok}[1]{\textcolor[rgb]{0.56,0.35,0.01}{\textbf{\textit{#1}}}}
\usepackage{longtable,booktabs}
% Fix footnotes in tables (requires footnote package)
\IfFileExists{footnote.sty}{\usepackage{footnote}\makesavenoteenv{longtable}}{}
\usepackage{graphicx,grffile}
\makeatletter
\def\maxwidth{\ifdim\Gin@nat@width>\linewidth\linewidth\else\Gin@nat@width\fi}
\def\maxheight{\ifdim\Gin@nat@height>\textheight\textheight\else\Gin@nat@height\fi}
\makeatother
% Scale images if necessary, so that they will not overflow the page
% margins by default, and it is still possible to overwrite the defaults
% using explicit options in \includegraphics[width, height, ...]{}
\setkeys{Gin}{width=\maxwidth,height=\maxheight,keepaspectratio}
\setlength{\emergencystretch}{3em}  % prevent overfull lines
\providecommand{\tightlist}{%
  \setlength{\itemsep}{0pt}\setlength{\parskip}{0pt}}
\setcounter{secnumdepth}{5}
% Redefines (sub)paragraphs to behave more like sections
\ifx\paragraph\undefined\else
\let\oldparagraph\paragraph
\renewcommand{\paragraph}[1]{\oldparagraph{#1}\mbox{}}
\fi
\ifx\subparagraph\undefined\else
\let\oldsubparagraph\subparagraph
\renewcommand{\subparagraph}[1]{\oldsubparagraph{#1}\mbox{}}
\fi

% set default figure placement to htbp
\makeatletter
\def\fps@figure{htbp}
\makeatother


\usepackage{amsmath}
\usepackage{booktabs}
\usepackage{caption}
\usepackage{longtable}
\usepackage[]{natbib}
\bibliographystyle{apalike}

\title{CorrelAid Data Journey}
\author{Yihui Xie}
\date{2020-06-09}

\begin{document}
\maketitle

{
\setcounter{tocdepth}{1}
\tableofcontents
}
\begin{Shaded}
\begin{Highlighting}[]
\KeywordTok{library}\NormalTok{(flextable)}
\KeywordTok{library}\NormalTok{(gt)}
\end{Highlighting}
\end{Shaded}

\hypertarget{einfuxfchrung}{%
\chapter{Einführung}\label{einfuxfchrung}}

\hypertarget{impact-measurement}{%
\chapter{Impact Measurement}\label{impact-measurement}}

\begin{Shaded}
\begin{Highlighting}[]
\CommentTok{# load data }
\NormalTok{drm <-}\StringTok{ }\NormalTok{readr}\OperatorTok{::}\KeywordTok{read_csv}\NormalTok{(}\StringTok{"data/datenreifegradmodell.csv"}\NormalTok{, }\DataTypeTok{quote =} \StringTok{"}\CharTok{\textbackslash{}"}\StringTok{"}\NormalTok{, }\DataTypeTok{na =} \KeywordTok{c}\NormalTok{(}\StringTok{""}\NormalTok{))}
\end{Highlighting}
\end{Shaded}

\begin{verbatim}
## Parsed with column specification:
## cols(
##   Thema = col_character(),
##   Unzureichend = col_character(),
##   Ausreichend = col_character(),
##   Fortgeschritten = col_character(),
##   Ausgezeichnet = col_character()
## )
\end{verbatim}

\begin{Shaded}
\begin{Highlighting}[]
\NormalTok{drm[}\KeywordTok{is.na}\NormalTok{(drm)] <-}\StringTok{ ""}
\end{Highlighting}
\end{Shaded}

You can write citations, too. For example, we are using the \textbf{bookdown} package in this sample book, which was built on top of R Markdown and \textbf{knitr} \citep{xie2015}.

\begin{Shaded}
\begin{Highlighting}[]
\CommentTok{# Move the time-based columns to the start of}
\CommentTok{# the column series; modify the column labels of}
\CommentTok{# the measurement-based columns}
\NormalTok{gt_tbl <-}\StringTok{ }\NormalTok{gt}\OperatorTok{::}\KeywordTok{gt}\NormalTok{(drm)}

\CommentTok{# Show the gt Table}
\NormalTok{gt_tbl <-}\StringTok{ }\NormalTok{gt_tbl }\OperatorTok\StringTok{ }
\StringTok{   }\KeywordTok{tab_header}\NormalTok{(}
    \DataTypeTok{title =} \KeywordTok{md}\NormalTok{(}\StringTok{"**Datenreifegrad sozialer Organisationen**"}\NormalTok{),}
    \DataTypeTok{subtitle =} \KeywordTok{md}\NormalTok{(}\StringTok{"Ein Modellierungsframework"}\NormalTok{)}
\NormalTok{  ) }\OperatorTok\StringTok{ }
\StringTok{  }\KeywordTok{tab_source_note}\NormalTok{(}
    \DataTypeTok{source_note =} \StringTok{"Quelle: Center for Data Science & Public Policy, Data Maturity Framework, University of Chicago."}
\NormalTok{  )}


\NormalTok{gt_tbl }\OperatorTok\StringTok{   }
\StringTok{  }\KeywordTok{tab_style}\NormalTok{(}
    \DataTypeTok{style =} \KeywordTok{list}\NormalTok{(}\KeywordTok{cell_fill}\NormalTok{(}\DataTypeTok{color =} \StringTok{"#d7191c"}\NormalTok{)),}
    \DataTypeTok{locations =} \KeywordTok{list}\NormalTok{(}\KeywordTok{cells_body}\NormalTok{(}\DataTypeTok{columns =} \KeywordTok{vars}\NormalTok{(Unzureichend)), }\KeywordTok{cells_column_labels}\NormalTok{(}\KeywordTok{vars}\NormalTok{(Unzureichend)))}
\NormalTok{    ) }\OperatorTok\StringTok{ }
\StringTok{  }\KeywordTok{tab_style}\NormalTok{(}
    \DataTypeTok{style =} \KeywordTok{list}\NormalTok{(}\KeywordTok{cell_fill}\NormalTok{(}\DataTypeTok{color =} \StringTok{"#fdae61"}\NormalTok{)),}
    \DataTypeTok{locations =} \KeywordTok{list}\NormalTok{(}\KeywordTok{cells_body}\NormalTok{(}\DataTypeTok{columns =} \KeywordTok{vars}\NormalTok{(Ausreichend)), }\KeywordTok{cells_column_labels}\NormalTok{(}\KeywordTok{vars}\NormalTok{(Ausreichend)))}
\NormalTok{  ) }\OperatorTok\StringTok{ }
\StringTok{  }\KeywordTok{tab_style}\NormalTok{(}
    \DataTypeTok{style =} \KeywordTok{list}\NormalTok{(}\KeywordTok{cell_fill}\NormalTok{(}\DataTypeTok{color =} \StringTok{"#a6d96a"}\NormalTok{)),}
    \DataTypeTok{locations =} \KeywordTok{list}\NormalTok{(}\KeywordTok{cells_body}\NormalTok{(}\DataTypeTok{columns =} \KeywordTok{vars}\NormalTok{(Fortgeschritten)), }\KeywordTok{cells_column_labels}\NormalTok{(}\KeywordTok{vars}\NormalTok{(Fortgeschritten)))}
\NormalTok{  ) }\OperatorTok\StringTok{ }
\StringTok{  }\KeywordTok{tab_style}\NormalTok{(}
    \DataTypeTok{style =} \KeywordTok{list}\NormalTok{(}\KeywordTok{cell_fill}\NormalTok{(}\DataTypeTok{color =} \StringTok{"#1a9641"}\NormalTok{)),}
    \DataTypeTok{locations =} \KeywordTok{list}\NormalTok{(}\KeywordTok{cells_body}\NormalTok{(}\DataTypeTok{columns =} \KeywordTok{vars}\NormalTok{(Ausgezeichnet)), }\KeywordTok{cells_column_labels}\NormalTok{(}\KeywordTok{vars}\NormalTok{(Ausgezeichnet)))}
\NormalTok{  )}
\end{Highlighting}
\end{Shaded}

\captionsetup[table]{labelformat=empty,skip=1pt}
\begin{longtable}{lllll}
\caption*{
\large \textbf{Datenreifegrad sozialer Organisationen}\\ 
\small Ein Modellierungsframework\\ 
} \\ 
\toprule
Thema & Unzureichend & Ausreichend & Fortgeschritten & Ausgezeichnet \\ 
\midrule
Zugang & Nur in der Anwendung verfügbar, in der die Daten erhoben werden & Daten können in einfachen Formaten (z.B. PDF) extrahiert werden & Daten können in maschinell lesbaren Formaten extrahiert werden (CSV, JSON, XML, Datenbankextrakt) &  \\ 
Integration & Papier &  &  &  \\ 
Relevanz und Vollständigkeit &  &  &  &  \\ 
Qualität & Daten sind in individuellen Anwendungen verfügbar &  &  &  \\ 
Erhebungsfrequenz & Zwischen tatsächlicher Datenlage und Datenbedarf liegt eine hohe Diskrepanz vor: Zum Beispiel fehlt bei der Analyse von Schulabbruchsquoten die Messgröße der Schulabbrecher:innen &  &  & Daten geben vollständig Auskunft über relevante Fragestellungen und können gezielt eingesetzt werden  um Lösungsvorschläge zu entwickeln \\ 
Granularität &  &  &  &  \\ 
Historie &  &  &  &  \\ 
Datenschutz &  &  &  &  \\ 
Dokumentation &  &  &  &  \\ 
\bottomrule
\end{longtable}
\begin{minipage}{\linewidth}
Quelle: Center for Data Science \& Public Policy, Data Maturity Framework, University of Chicago.\\ 
\end{minipage}

\hypertarget{references}{%
\chapter*{References}\label{references}}
\addcontentsline{toc}{chapter}{References}

\bibliography{book.bib}

\end{document}
